\chapter*{Resumen Ejecutivo}
\thispagestyle{first-pages}
En la actualidad los sistemas de seguridad cubren un abanico de herramientas que permiten reducir riesgos y prever infortunios. Estos sistemas están compuestos tanto por herramientas de hardware, como de software, los cuales han evolucionado a lo largo de los años de manera acelerada.

El software hoy en día puede aprovechar al máximo los recursos de una infraestructura de una empresa mediante la construcción de aplicaciones que hacen uso de los principios de una arquitectura orientada a servicios, también llamadas aplicaciones nativas de la nube, y basadas en conceptos como fiabilidad, tolerancia a fallos y escalabilidad. 

En este sentido, el software de visión artificial enfocado en casos de uso del área de sistemas de seguridad es capaz agregar valor a las empresas y de reducir costos de personal innecesarios. Una manera de atacar esta problemática es aprovechar los componentes que disponen las empresas, como las cámaras de seguridad y mediante tecnologías de visión artificial, para identificar las matriculas de los vehículos que ingresan o salen de un perímetro, bajo condiciones controladas.


\renewcommand{\baselinestretch}{1.6} 
\pagebreak