\chapter{Implementación}
Dado el nivel de las tecnologías descrito en los capítulos II-VI, el PFG utiliza:
\textbf{Tecnologías del Negocio}
\begin{itemize}
    \item OpenCV para la implementación del servicio recolector.
    \item OpenALPR para el reconocimiento de matriculas. Esta libreria ya que utiliza OpenCV que es la tecnología principal en computación visual, además de ser open-source, multiplataforma y ser utilizada por compañías como Microsoft, IBM, Intel, etc. \cite{Itseez2000-he}.
\end{itemize}

\textbf{Tecnologías de Soporte}
\begin{itemize}
    \item Kubernetes para la gestión y configuración de contendedores, dados los casos de uso exitosos en producción en compañías como Google, SAP, Ebay, Box \cite{Kubernetes2016-ub}; y el poder de ser ejecutado en cualquier nube.
    
    \item Laravel como Framework de desarrollo web, elegido en base a la experiencia del equipo de programadores con esta herramienta, la documentación que provee Laravel y las capacidades que posee, descritas en el Capitulo 5.
        
    \item Las cámaras de marca Dahua son las que utiliza QSS Bolivia y con ellas se realiza las pruebas del reconocimiento. Las especificaciones se describen en el Apéndice \ref{appendix:dahua}.

    \item PostgreSQL es la base de datos de preferencia, dada la experiencia del grupo de programadores con esta herramienta, y por ser open-source.
\end{itemize}
    