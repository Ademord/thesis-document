\chapter{Implementación}
Dado el nivel actual de las tecnologías descrito en los capítulos III-VI, para la implementación del PFG se utilizó las herramientas descritas a continuación.

\section{Tecnologías del Negocio}
\begin{itemize}
    \item La capa de presentación se programó con HTML, CSS y PHP en armonía con el Framework de desarrollo web Laravel. Se utilizo plantillas de Bootstrap\footnote{\url{http://getbootstrap.com/}} para el diseño de botones, tablas, etc.
    \item Los servicios Lugar, Cámara, Matricula, Propietario, Coincidencia, Recolector y Reconocedor se programaron en Python 3.5. 
    \item La interfaz que expone cada servicio y se mapea a un método del mismo se programó con la librería Flask
     \footnote{\url{http://flask.pocoo.org/}}.
    \item El Servicio Recolector recupera la entrada de vídeo a través de las funcionalidades de la librería OpenCV.
    \item El Servicio Reconocedor utiliza OpenALPR para el reconocimiento de matriculas. Esta librería utiliza OpenCV de manera subyacente. OpenCV es la tecnología principal en computación visual, además de ser open-source, multiplataforma y ser utilizada por compañías como Microsoft, IBM, Intel, etc. \cite{Itseez2000-he}.       
\end{itemize}

\section{Tecnologías de Soporte}
\begin{itemize}
    \item Kubernetes para la gestión y configuración de clusteres de contendedores, dados los casos de uso exitosos en producción en compañías como Google, SAP, Ebay, Box \cite{Kubernetes2016-ub}; y el poder de ser ejecutado en cualquier nube.
    
    \item Laravel como Framework de desarrollo web, elegido en base a la experiencia del equipo de programadores con esta herramienta, la documentación que provee Laravel y las capacidades que posee, descritas en el Capitulo 5.
        
    \item Las cámaras de marca Dahua son las que utiliza QSS Bolivia y con ellas se realiza las pruebas del reconocimiento. Las especificaciones se describen en el Apéndice \ref{appendix:dahua}.

    \item PostgreSQL es la base de datos de preferencia, dada la experiencia del grupo de programadores con esta herramienta, y por ser open-source.
\end{itemize}
\section{Servicio}
\subsection{Descripción}
En breve, para programar un servicio se debe:
\begin{itemize}
    \item Programar la interfaz del servicio y la lógica del mismo.
    \item Definir la imagen Docker.
    \item Definir el archivo de requerimientos.
    \item Definir el archivo de despliegue en Kubernetes.
    \item Definir el archivo de Servicio Kubernetes.
\end{itemize}

Dado lo anterior, la estructura de archivos de un servicio consiste en la siguiente:

{\setstretch{1.0}
\vspace{0.5cm}
\dirtree{%
.1 servicio/.
    .2 app/.
        .3 app.py.
        .3 database.py.
        .3 Dockerfile.
        .3 requirements.txt. 
        .3 server.py.
    .2 deployment.json.
    .2 service.json.
}
}
\pagebreak
Donde:
\begin{itemize}
\item \textit{app.py} contiene la lógica del servicio.
\item \textit{database.py} contiene la configuración de la base de datos.
\item \textit{Dockerfile} describe la implementación del servicio en un contenedor Docker. Este archivo extiende una imagen base de DockerHub  \footnote{\url{https://hub.docker.com/}} que incluye una versión mínima de Linux y las dependencias base para la implementación del servicio. Esta imagen es extendida de acuerdo a la aplicación que se desarrolla. Por ejemplo, los servicios atómicos se basan en una imagen que únicamente incluye python3.5, mientras que el servicio recolector se basa en una imagen de OpenCV y el servicio reconocedor en una imagen de OpenALPR.
\item \textit{requirements.txt} es un archivo que contiene todas las dependencias.
\item \textit{server.py} es el encargado de ejecutar el micro servidor web para mapear las rutas 
\item \textit{deployment.json} describe los metadatos necesarios para el despliegue del servicio en Kubernetes (cantidad de replicas, variables de entorno, puertos, etc.).
\item \textit{service.json} describe el Servicio Kubernetes de tipo Balanceador de Carga que permite accesos al servicio desde el exterior.


\end{itemize}

\subsection{Implementación}
Los servicios, al ser aislados e independientes, exponen una interfaz con la cual otros componentes pueden comunicarse con ellos. Esta interfaz se programó con Flask, que monta un micro servidor web que mapea las rutas permitidas a un método del servicio. Un ejemplo de un servicio es:

{ \sffamily \textbf{app.py}}

\begin{mdframed}[linecolor=black, topline=true, bottomline=true,
  leftline=false, rightline=false, backgroundcolor=LightGray,userdefinedwidth=\textwidth]
  \begin{minted}[fontsize=\normalsize, linenos, baselinestretch=1.2]{python}

from orator import Model
from database import db

class Lugar(Model):

    def connect():
        Model.set_connection_resolver(db)
    __table__ = 'lugar'
    
    def getall(): 
        return Lugar.all().serialize()

    def get(id):
        lugar = Lugar.find(id)
        try:
            return lugar.serialize()
        except:
            return lugar

    def add(temp):
        lugar = Lugar()
        try:
            lugar.nombre = temp['nombre']
            lugar.save()
            return True
        except:
            return "No se pudo agregar elemento."

    def remove(id):
        lugar = Lugar.find(id)
        try:
            lugar.delete()
            return True
        except:
            return "No se pudo eliminar elemento."

    def update(id, temp):
        lugar = Lugar.find(id)
        try:
            lugar.nombre = temp['nombre']
            lugar.save()
            return True
        except:
            return "No se pudo actualizar elemento."

    def valid(temp):
        if 'nombre' in temp and not isinstance(temp['nombre'], str):
            return False
        return True
\end{minted}
\end{mdframed}

\pagebreak
{ \sffamily \textbf{database.py}} 

\begin{mdframed}[linecolor=black, topline=true, bottomline=true,
  leftline=false, rightline=false, backgroundcolor=LightGray,userdefinedwidth=\textwidth]
  \begin{minted}[fontsize=\normalsize, linenos, baselinestretch=1.2]{python}

from orator import DatabaseManager
import os
config = {
    'pgsql': {
        'driver': 'pgsql',
        'host': os.environ.get('DATABASE_HOST', 'homestead'),
        'database': os.environ.get('DATABASE_NAME', 'homestead'),
        'user': os.environ.get('DATABASE_USER', 'homestead'),
        'password': os.environ.get('DATABASE_PASSWORD', 'secret'),
        'prefix': '',
        'port': os.environ.get('DATABASE_PORT', '5432')
    }
}
db = DatabaseManager(config)

\end{minted}
\end{mdframed}
\vspace{0.2cm}
{ \sffamily 
\textbf{Dockerfile}
}\vspace{0.1cm}


\begin{mdframed}[linecolor=black, topline=true, bottomline=true,
  leftline=false, rightline=false, backgroundcolor=LightGray,userdefinedwidth=\textwidth]
  \begin{minted}[fontsize=\normalsize, linenos, baselinestretch=1.2]{vim}
FROM python:3-onbuild
ENV DATABASE_HOST=homestead
ENV DATABASE_NAME=homestead
ENV DATABASE_USER=homestead
ENV DATABASE_PASSWORD=secret
ENV DATABASE_PORT=5432
EXPOSE 55550
ENTRYPOINT ["python","/usr/src/app/server.py"]
\end{minted}
\end{mdframed}

{ \sffamily 
\textbf{requirements.txt}
}
\begin{mdframed}[linecolor=black, topline=true, bottomline=true,
  leftline=false, rightline=false, backgroundcolor=LightGray,userdefinedwidth=\textwidth]
  \begin{minted}[fontsize=\normalsize, linenos, baselinestretch=1.2]{python}
flask
orator
psycopg2
\end{minted}
\end{mdframed}

\pagebreak
{ \sffamily \textbf{server.py} }


\begin{mdframed}[linecolor=black, topline=true, bottomline=true,
  leftline=false, rightline=false, backgroundcolor=LightGray,userdefinedwidth=\textwidth]
  \begin{minted}[fontsize=\normalsize, linenos, baselinestretch=1.2]{python}

from flask import Flask, jsonify, abort, make_response, request
from app import Lugar
app = Flask(__name__)

@app.route('/lugares', methods=['GET'])
def index():
    temp = Lugar.getall()
    resultado = {}
    resultado['lugares'] = temp
    return jsonify(resultado)

@app.route('/lugares/<int:id>', methods=['GET'])
def show(id):
    temp = Lugar.get(id)
    if not temp: abort(404)
    resultado = {}
    resultado['lugar'] = temp
    return jsonify(resultado)

@app.route('/lugares', methods=['POST'])
def store():
    data = request.get_json()
    if not request.json or not Lugar.valid(data):
        abort(400)
    result = Lugar.add(data)
    return jsonify({'result': result}), 201

@app.route('/lugares/<int:id>', methods=['PUT'])
def update(id):
    temp = Lugar.get(id)
    data = request.get_json()
    if not temp or not data or not Lugar.valid(data):
        abort(404)
    result = Lugar.update(id, data)
    return jsonify({'result': result }), 201

@app.route('/lugares/<int:id>', methods=['DELETE'])
def destroy(id):
    temp = Lugar.get(id)
    if not temp: abort(404)
    result = Lugar.remove(id)
    return jsonify({'result': result }), 201

@app.errorhandler(404)
def not_found(error):
    return make_response(jsonify({'error': 'Not found'}), 404)

@app.before_first_request
def _run_on_start():
    Lugar.connect()

if __name__ == '__main__':
    app.run(host='0.0.0.0', port=55550, debug=True)

\end{minted}
\end{mdframed}
\pagebreak
{ \sffamily \textbf{deployment.json} }
\begin{mdframed}[linecolor=black, topline=true, bottomline=true,
  leftline=false, rightline=false, backgroundcolor=LightGray,userdefinedwidth=\textwidth]
  \begin{minted}[fontsize=\normalsize, linenos, baselinestretch=1]{json}
{  
   "kind":"Deployment",
   "apiVersion":"extensions/v1beta1",
   "metadata":{  
      "name":"lugares",
      "namespace":"panchito",
      "creationTimestamp":null
   },
   "spec":{  
      "replicas":3,
      "template":{  
         "metadata":{  
            "creationTimestamp":null,
            "labels":{  
               "service":"lugares"
            }
         },
         "spec":{  
            "containers":[  
               {  
                  "name":"lugares",
                  "image":"ademord/lugar:v6",
                  "ports":[  
                     {  
                        "containerPort":55550,
                        "protocol":"TCP"
                     }
                  ],
                  "env":[  
                     {  
                        "name":"DATABASE_HOST",
                        "value":"homestead"
                     },
                     {  
                        "name":"DATABASE_PORT",
                        "value":"5432"
                     }
                  ],
                  "resources":{ }
               }
            ],
            "restartPolicy":"Always"
         }
      },
      "strategy":{ }
   },
   "status":{ }
}
\end{minted}
\end{mdframed}

{ \sffamily \textbf{service.json} }
\begin{mdframed}[linecolor=black, topline=true, bottomline=true,
  leftline=false, rightline=false, backgroundcolor=LightGray,userdefinedwidth=\textwidth]
  \begin{minted}[fontsize=\normalsize, linenos, baselinestretch=1]{json}
 {  
   "kind":"Service",
   "apiVersion":"v1",
   "metadata":{  
      "name":"lugares",
      "namespace":"panchito",
      "creationTimestamp":null,
      "labels":{  
         "service":"lugares"
      }
   },
   "spec":{  
      "ports":[  
         {  
            "name":"55550",
            "protocol":"TCP",
            "port":80,
            "targetPort":55550
         }
      ],
      "selector":{  
         "service":"lugares"
      },
      "type":"LoadBalancer"
   },
   "status":{  
      "loadBalancer":{ }
   }
}
\end{minted}
\end{mdframed}

Con estos archivos se puede montar un servicio en un cluster de Kubernetes\footnote{\url{http://kubernetes.io/docs/hellonode/}}. Cabe recordar que existen diversas alternativas de Infraestructura como Servicio (IaaS) en el momento de utilizar Kubernetes\footnote{\url{http://kubernetes.io/docs/getting-started-guides/}}:
\begin{itemize}
\item \textbf{Nubes Públicas:} contratar un proveedor de IaaS y montar el cluster de Kubernetes sobre sus recursos de cómputo, almacenamiento, redes, etc. como AWS, Google Cloud, etc.
\item \textbf{Nubes Privadas:} alojar la aplicación en servidores físicos en las instalaciones de la empresa.
\item \textbf{Nubes Híbridas:} mantener cierta parte de la infraestructura in-situ (en las instalaciones), donde la estrategia general es desplegar los nodos de aplicación en la nube y mantener las bases de datos en los datacenter de la empresa.
 \end{itemize}
A pesar de estas alternativas, posterior a pruebas en despliegues en nubes públicas, se registró una latencia de 4 a 12 segundos en el envío de cuadros de vídeo a través de Internet. Esta latencia se encuentra presente debido a la infraestructura del internet de Bolivia y, se traduce en un requerimiento de desplegar la aplicación en una nube privada, donde no existe esta condicionante.

\section{Recolección, Reconocimiento y Almacenamiento}
\subsection{Descripción}

El Servicio Recolector, manipula los cuadros de vídeo en una estructura particular llamada arreglos de numpy. Estos arreglos son transformados a listas y enviados, con la IP de la cámara correspondiente, en formato JSON a través de HTTP a la interfaz correspondiente del Servicio Reconocedor.

El servicio Reconocedor procesa cada cuadro y si encuentra una matricula la envía, con la IP de la cámara correspondiente, al Servicio de Coincidencias el cual se encarga de almacenar la matricula de acuerdo a la descripción del diagrama de secuencia \ref{fig:seq-alma}.

\subsection{Implementación}

Inicialmente, el servicio recolector y el servicio reconocedor fueron implementados con una arquitectura monolítica. Posterior a pruebas, se observó que la cantidad de cuadros generados saturaba al modulo de reconocimiento, por lo que se intento utilizar hilos para tener procesamiento en paralelo y solventar el problema generado por la tasa de servicio. Ante esto, se observó deficiencias en este nuevo modelo de proceso, debido a que:
\pagebreak
\begin{enumerate}
\item no se podía utilizar el patrón de mensajería Dispersión\footnote{\url{http://www.ibm.com/support/knowledgecenter/SSCGGQ_1.2.0/com.ibm.ism.doc/Planning/ov00101_.html}} (en ingles, Fan Out), que permite el envió de un mensaje a múltiples destinos para realizar procesamiento en paralelo, sin implementaciones tediosas y robustas en conocimiento en patrones de hilos, 
\item se observo que al ejecutar una solución que ejecute hilos, se obtenía un único proceso python que manejaba hilos donde cada uno lee cuadros de vídeo en mediana-alta calidad, los redirecciona y posteriormente los procesa para reconocer matriculas en los mismos. Esto incurría en un proceso python que utilizaban memoria ineficientemente principalmente debido al:
    \begin{itemize}
    \item \textit{overhead} generado en la ejecución de un solo proceso que a su vez ejecutaba hilos 
    \item manejo de memoria compartida a través de hilos que realizan procesamiento costoso 
    \end{itemize}
\end{enumerate}

La implementación del Servicio Recolector y el Servicio Reconocedor con contenedores permitió solventar este problema, debido a que capacidad de escalar la aplicación elimina la propuesta de utilización de hilos en una arquitectura monolítica. 
Más aun, la implementación mediante contenedores permitió aislar los errores encontrados, desplegar sobre cualquier plataforma y maximizar desempeño optimizando el uso de recursos. A continuación se adjuntan el codigo fuente involucrado en la implementación de cada servicio, detallando un poco mas en el Servicio Reconocedor (archivos .json).
% \pagebreak
\subsubsection{Servicio Recolector}
{\sffamily \textbf{server.py}}
\begin{mdframed}[linecolor=black, topline=true, bottomline=true,
  leftline=false, rightline=false, backgroundcolor=LightGray,userdefinedwidth=\textwidth]
\begin{minted}[fontsize=\normalsize, linenos, baselinestretch=1]{python}
#!flask/bin/python
from flask import Flask, jsonify, make_response, request

import cv2
import numpy as np

import requests
import json
import os
from datetime import datetime

app = Flask(__name__)

config ={
    "ip": "123.123.111"    
}

@app.route('/', methods=['GET'])
def status():
    return jsonify(config)
@app.route('/<string:ip>', methods=['POST'])
def set_ip(ip):
    config['ip'] = ip
    return jsonify(config)


def send(data):
    data ['timestamp'] = str(datetime.now())
    url = os.environ.get('TARGET_SERVICE') + "/reconocer"
    headers = {'Content-Type': 'application/json'}
    r = requests.
        post(url.strip(), data=json.dumps(data), headers=headers)
    return jsonify({'result': 'sent to ' + url}), 201
# <string:URI>

@app.route('/read', methods=['GET'])
def read():
    img = cv2.imread('image.jpg', 0)
    return send({'ip': config['ip'], 'frame': img.tolist()})

@app.errorhandler(404)
def not_found(error):
    return make_response(jsonify({'error': 'Not found'}), 404)

if __name__ == '__main__':
    app.run(host='0.0.0.0', port=55551, debug=True)

\end{minted}
\end{mdframed}

\subsubsection{Servicio Reconocedor}
{\sffamily \textbf{requirements.txt}}
\begin{mdframed}[linecolor=black, topline=true, bottomline=true,
  leftline=false, rightline=false, backgroundcolor=LightGray,userdefinedwidth=\textwidth]
\begin{minted}[fontsize=\normalsize, linenos, baselinestretch=1]{vim}
flask
request
numpy
argparse
\end{minted}
\end{mdframed}

{ \sffamily \textbf{Dockerfile}}
\vspace{0.1cm}
\begin{mdframed}[linecolor=black, topline=true, bottomline=true,
  leftline=false, rightline=false, backgroundcolor=LightGray,userdefinedwidth=\textwidth]
\begin{minted}[fontsize=\normalsize, linenos, baselinestretch=1]{vim}
FROM tsutomu7/python-opencv
USER root
# Install prerequisites
run apt-get update && apt-get install -y \
    build-essential \
    cmake \
    curl \
    git \
    libcurl3-dev \
    libleptonica-dev \
    liblog4cplus-dev \
    libopencv-dev \
    libtesseract-dev \
    wget
run apt-get update && sudo apt-get install -y \
    openalpr openalpr-daemon openalpr-utils libopenalpr-dev
EXPOSE 5000
RUN apt-get update
RUN mkdir -p /usr/src/app
WORKDIR /usr/src/app
COPY requirements.txt /usr/src/app/
RUN pip install --no-cache-dir -r requirements.txt
COPY . /usr/src/app
ENTRYPOINT ["python","/usr/src/app/server.py"]
\end{minted}
\end{mdframed}

{\sffamily \textbf{server.py}}
\begin{mdframed}[linecolor=black, topline=true, bottomline=true,
  leftline=false, rightline=false, backgroundcolor=LightGray,userdefinedwidth=\textwidth]
\begin{minted}[fontsize=\normalsize, linenos, baselinestretch=1]{python}
#!flask/bin/python
from flask import Flask, abort, jsonify, request
from subprocess import check_output
import subprocess
app = Flask(__name__)
import numpy as np
import sys
import io
import re
import functools
import cv2

URL = "/usr/src/app/image.jpg"
logs = [{
    "inicio": "mensaje inicial"	
}]

@app.route('/reconocer', methods=['POST'])
def _reconocer():
    logs.append({'reconocer':'hola'})
    # reconocer = compose(forward, clean, format, process, save)
    results = 
        forward(
            clean(
                format(
                    process(
                        save(request.get_json()
        )))))
    return jsonify({'logs': logs})

# @app.route('/save', methods=['POST'])
def save(data):
    logs.append({'save':'hola'})
    frame = np.array(data['frame'])
    cv2.imwrite(URL, frame); 
    #test image is saved
    # return jsonify({'logs': logs})
    return data

# @app.route('/process', methods=['GET'])
def process(data):
    logs.append({'process':'hola'})
    # open frame with command line
    # test alpr works, by bashing into container first
    args = ["alpr", URL, "-c", "bo","--config",
        "/usr/src/app/openalpr.conf", "-p","bo", "--clock"]
    data['output'] = subprocess.Popen(args, stdout=subprocess.PIPE)
    return data

def format(data):
    logs.append({'format':'hola'})
    lines = []
    regex = re.compile(r'[\n\r\t]')
    for line in io.TextIOWrapper(data['output'].
        stdout, encoding="utf-8"):
        if not 'confidence' in line:
            continue
        line = regex.sub(' ', line)
        plate = re.
            findall ( '    - (.*?)  confidence', line, re.DOTALL)[0]
        confidence = 
            re.findall ( 'confidence: (.*?) ', line, re.DOTALL)[0]
        # logs.append({"plate found": line})
        lines.
            append({"plate found": plate, "confidence": confidence})
    try:
        first_plate = lines[0]
    except:
        first_plate = lines
    del data['output']
    data['plate'] = first_plate
    return data
    # return jsonify({'logs': logs[0]})

# @app.route('/delete', methods=['GET'])
def clean(data):
    logs.append({'clean':'hola'})
    # # destroy frame
    args = ["rm", URL]
    subprocess.Popen(args, stdout=subprocess.PIPE)
    # #test image is deleted
    # return jsonify("Deleted")
    return data

def forward(data):
    del data['frame']
    logs.append({'data': data})
    return jsonify({'data': data})
	

@app.route('/', methods=['GET'])
def index():
logs.append({'hola': 'no he recibido nada'})
return jsonify({'logs': logs})

if __name__ == '__main__':
    app.run(host='0.0.0.0', port=5000, debug=True)

\end{minted}
\end{mdframed}
\pagebreak
{\sffamily \textbf{deployment.json}}
\begin{mdframed}[linecolor=black, topline=true, bottomline=true,
  leftline=false, rightline=false, backgroundcolor=LightGray,userdefinedwidth=\textwidth]
\begin{minted}[fontsize=\normalsize, linenos, baselinestretch=1]{json}
{  
   "kind":"Deployment",
   "apiVersion":"extensions/v1beta1",
   "metadata":{  
      "name":"reconocedor",
      "namespace":"panchito",
      "creationTimestamp":null
   },
   "spec":{  
      "replicas":1,
      "template":{  
         "metadata":{  
            "creationTimestamp":null,
            "labels":{  
               "service":"reconocedor"
            }
         },
         "spec":{  
            "containers":[  
               {  
                  "name":"reconocedor",
                  "image":"ademord/reconocedor:latest",
                  "ports":[  
                     {  
                        "containerPort":5000,
                        "protocol":"TCP"
                     }
                  ],
                  "resources":{ }
               }
            ],
            "restartPolicy":"Always"
         }
      },
      "strategy":{ }
   },
   "status":{ }
}
\end{minted}
\end{mdframed}

{\sffamily \textbf{service.json}}
\begin{mdframed}[linecolor=black, topline=true, bottomline=true,
  leftline=false, rightline=false, backgroundcolor=LightGray,userdefinedwidth=\textwidth]
\begin{minted}[fontsize=\normalsize, linenos, baselinestretch=1]{json}
{  
   "kind":"Service",
   "apiVersion":"v1",
   "metadata":{  
      "name":"reconocedor",
      "namespace":"panchito",
      "creationTimestamp":null,
      "labels":{  
         "service":"reconocedor"
      }
   },
   "spec":{  
      "ports":[  
         {  
            "name":"5000",
            "protocol":"TCP",
            "port":80,
            "targetPort":5000
         }
      ],
      "selector":{  
         "service":"reconocedor"
      },
      "type":"LoadBalancer"
   },
   "status":{  
      "loadBalancer":{ }
   }
}
\end{minted}
\end{mdframed}

{\sffamily \textbf{redeploy.bash}}
\begin{mdframed}[linecolor=black, topline=true, bottomline=true,
  leftline=false, rightline=false, backgroundcolor=LightGray,userdefinedwidth=\textwidth]
\begin{minted}[fontsize=\normalsize, linenos, baselinestretch=1]{vim}
kubectl delete deployment reconocedor && \
    docker build -t ademord/reconocedor:latest ./app && \
        docker push ademord/reconocedor:latest && \
            kubectl create -f deployment.json

\end{minted}
\end{mdframed}

\subsubsection{Servicio de Coincidencias}
{\sffamily \textbf{server.py}}
\begin{mdframed}[linecolor=black, topline=true, bottomline=true,
  leftline=false, rightline=false, backgroundcolor=LightGray,userdefinedwidth=\textwidth]
\begin{minted}[fontsize=\normalsize, linenos, baselinestretch=1]{python}
#!flask/bin/python
from flask import Flask, jsonify, abort, make_response, request
from app import Coincidencia
app = Flask(__name__)
logs = [{
    "inicial":"init"
}
]
@app.route('/', methods=['GET'])
def index():
    return jsonify(Coincidencia.getall())

@app.route('/<int:id>', methods=['GET'])
def show(id):
     return jsonify(Coincidencia.get(id))

@app.route('/search/<string:q>', methods=['GET'])
def search(q):
    return jsonify(Coincidencia.buscar(q))

@app.route('/', methods=['POST'])
def store():
    data = request.get_json()
    if not request.json or not Coincidencia.valid(data):
        abort(400)
    return jsonify(Coincidencia.add(data)), 201

@app.route('/<int:id>', methods=['DELETE'])
def destroy(id):
    temp = Coincidencia.get(id)
    if not temp: abort(404)
    result = Coincidencia.remove(id)
    return jsonify({'result': result }), 201

@app.errorhandler(404)
def not_found(error):
    return make_response(jsonify({'error': 'Not found'}), 404)

@app.before_first_request
def _run_on_start():
    Coincidencia.connect()

if __name__ == '__main__':
    app.run(host='0.0.0.0', port=55555, debug=True)
    # app.run(port=55555, debug=True)

\end{minted}
\end{mdframed}

{\sffamily \textbf{app.py}}
\begin{mdframed}[linecolor=black, topline=true, bottomline=true,
  leftline=false, rightline=false, backgroundcolor=LightGray,userdefinedwidth=\textwidth]
\begin{minted}[fontsize=\normalsize, linenos, baselinestretch=1]{python}
from orator import Model
from database import db
from flask import abort
import uuid
import re
import json
import requests
import cv2
import numpy as np

#orator.orm.collection.Collection object at 0x000001AA82700978
class Coincidencia(Model):
    def connect():
        Model.set_connection_resolver(db)
    __table__ = 'coincidencias'

    def get_external(url):
        r = requests.get(url.strip())
        return r.json()

    def getall(): 
        try:
            return Coincidencia.all().serialize()
        except:
            return abort(404)

    def get(id):
        try:
            return Coincidencia.find(id).serialize()
        except:
            return abort(404)

    def buscar(param):
        try:
            return Coincidencia.where('matricula', 'like',
                '%{}%'.format(param)).get().serialize();
        except:
            return abort(404)

    def add(data):
        coincidencia = Coincidencia()
        # IP
        coincidencia.camara = data['ip']
        # LUGAR
        lugar_id = Coincidencia.
            get_external('http://camaras/search/{}'.
                format(data['ip']))[0]['lugar_id']
        lugar = Coincidencia.
            get_external('http://lugares/{}'.format(lugar_id))
        coincidencia.lugar = lugar['nombre']
        # MATRICULA
        coincidencia.matricula = data['plate']
        pattern = re.compile(r"^\d{1,4}[A-Z]{3}$",re.I)
        mismatch = True
        if pattern.search(coincidencia.matricula): mismatch = False
        coincidencia.mismatch = mismatch
        # PROPIETARIO
        coincidencia.propietario = "Desconocido" 
        #se asume que es desconocido
        if not mismatch:
            id_propietario = Coincidencia.
                get_external('http://matriculas/search/{}'.
                    format(coincidencia.matricula))
            if len(id_propietario):
                id_propietario = id_propietario[0]['propietario_id']
                propietario = Coincidencia.
                    get_external('http://propietarios/{}'.
                        format(id_propietario))
                coincidencia.propietario = propietario['nombres'] 
                    + propietario['apellidos']
                
        # CUADRO
        filename = str(uuid.uuid4()) + ".png"
        coincidencia.filename = filename
        coincidencia.mime = "image/png"
        cv2.imwrite('/usr/src/data/' + filename,
            np.array(data['frame'])); 
        del data['frame']

        coincidencia.save()
        return 200

    def remove(id):
        try:
            Coincidencia.find(id).delete()
            return 200
        except:
            return "No se pudo eliminar elemento."

    def valid(temp):
        if 'ip' in temp and 
        not isinstance(temp['plate'], str): #cambiar
            return False
        return True
\end{minted}
\end{mdframed}

El resto del código fuente para el PFG puede encontrarse en Github\footnote{\url{https://github.com/Ademord/control-vehicular-SOA}}.