\chapter{Recomendaciones}
A partir de la experiencia obtenida a lo largo del desarrollo del PFG, se adquirieron muchos conocimientos en áreas de investigación que no suelen ser mencionadas en Santa Cruz de la Sierra. En este sentido, cabe resaltar la importancia de apoyar la investigación en cada uno de nosotros para conocer el estado actual de las tecnologías a nivel mundial. Mediante una analogía, recomiendo no quedarse con un celular polifónico mientras en otros países en el mundo se utilizan teléfonos inteligentes.

En lo que respecta al PFG, como futuros pasos, primero, se recomienda realizar un entrenamiento personalizado para el clasificador, para matrículas bolivianas, ya que tener un clasificador entrenado en matrículas de la ubicación exacta, en cualquier prueba, generará resultados con mayor precisión que utilizando matrículas de otra región (a menos que se pueda reutilizar un entrenador de otra región cuya distribución alfanumérica de caracteres de matrículas sea idéntica que la local). 

Segundo, se recomienda profundizar en conceptos de computación en la nube, SOA, etc., con el fin de tener mayor conocimiento acerca de las herramientas que se está utilizando, sus beneficios y, poder realizar optimizaciones o solucionar problemas cuando convenga. 

Tercero, se recomienda considerar el desarrollo de un módulo de identificación de situaciones sospechosas “a prueba de balas”, que consista en detectar tanto vehículos como reconocer matrículas, a partir de los cuadros de video leídos. Esto permitiría alertar sobre situaciones sospechosas en donde 1) se observe un vehículo sin matrícula y 2) se reconozca una matrícula y no un vehículo; las cuales pueden ser posteriormente revisadas por un agente humano.

Finalmente, se recomienda considerar el desarrollo de un módulo de post-procesamiento, que coleccione las matrículas reconocidas distinguiendo cuales pertenecen a un vehículo y, a partir de un análisis por histogramas, optimice el criterio de identificación de la cadena de caracteres más probable que corresponda a la matrícula del vehículo.

Es en nuestro tiempo que, como ingenieros de sistemas, considero importante y recomiendo, dedicarse a la investigación, además de impulsar la educación tanto en nuevas tecnologías como en fortalecer los conceptos básicos y cimientos de la educación. 
\vspace*{\fill}
\epigraph{\itshape La inteligencia sin ambición es un pájaro sin alas}{---Salvador Dalí} 